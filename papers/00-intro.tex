% 00-intro.tex
\documentclass[11pt]{article}
\usepackage{amsmath,amssymb}
\usepackage{graphicx}
\usepackage{hyperref}

\begin{document}

\section*{Introduction to the Unified LQG Framework}

\subsection*{Theoretical Foundations}
This collection presents a comprehensive framework for Loop Quantum Gravity (LQG) applications to exotic spacetime physics, with particular emphasis on warp drive feasibility analysis. The unified approach combines canonical LQG quantization with advanced numerical methods for practical phenomenological calculations.

\subsection*{Numerical Methods Survey}
The framework employs several sophisticated computational techniques:
\begin{itemize}
  \item Adaptive mesh refinement with quantum geometry resonance detection
  \item Constraint algebra implementation with polymer modifications
  \item Multi-parameter optimization using evolutionary algorithms
  \item High-performance computing with MPI/GPU acceleration
  \item Advanced error estimation and convergence analysis
\end{itemize}

We also develop the \textbf{LQG-ANEC Framework}, a unified Python-based toolkit for computing Averaged Null Energy Condition violations in LQG coherent states, polymer quantization, and EFT settings (see the LQG-ANEC Framework section).

\subsection*{Organization}
The papers are organized thematically, beginning with fundamental LQG modifications to quantum field theory, proceeding through numerical implementation details, and culminating in applications to exotic matter engineering and warp drive physics.

\end{document}
