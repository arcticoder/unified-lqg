% drude_model.tex
\documentclass[11pt]{article}
\usepackage{amsmath,amssymb}
\usepackage{graphicx}
\usepackage{hyperref}
\usepackage{booktabs}

\begin{document}

\section*{Drude-Lorentz Permittivity Models for Vacuum Engineering}

\subsection*{Overview}
The Drude-Lorentz model provides a comprehensive framework for characterizing the dielectric response of materials used in vacuum engineering applications. This model combines free-electron behavior (Drude model) with bound-electron oscillator responses (Lorentz model) to accurately describe the frequency-dependent permittivity across the electromagnetic spectrum.

\subsection*{Mathematical Framework}
The complex permittivity is given by:
\[
  \varepsilon(\omega) = \varepsilon_\infty + \frac{\omega_p^2}{\omega_0^2 - \omega^2 - i\gamma\omega} - \frac{\omega_p^2}{\omega^2 + i\gamma\omega}
\]
where $\varepsilon_\infty$ is the high-frequency permittivity, $\omega_p$ is the plasma frequency, $\omega_0$ is the oscillator frequency, and $\gamma$ is the damping coefficient.

\subsection*{Materials \& Methods}

\subsubsection*{Drude–Lorentz Permittivity Selection}
Our material characterization employs a systematic approach to Drude-Lorentz parameter determination:

\begin{itemize}
  \item \textbf{Noble Metals (Au, Ag, Cu):} Parameters optimized for optical and near-infrared frequencies
    \begin{itemize}
      \item Gold: $\omega_p = 1.37 \times 10^{16}$ rad/s, $\gamma = 1.05 \times 10^{14}$ rad/s
      \item Silver: $\omega_p = 1.39 \times 10^{16}$ rad/s, $\gamma = 3.23 \times 10^{13}$ rad/s
      \item Copper: $\omega_p = 1.64 \times 10^{16}$ rad/s, $\gamma = 4.08 \times 10^{13}$ rad/s
    \end{itemize}
  
  \item \textbf{Dielectrics (SiO₂, Si₃N₄):} Multi-oscillator Lorentz models for transparency windows
    \begin{itemize}
      \item SiO₂: Three-oscillator model with $\omega_0 = \{2.75, 15.8, 23.0\} \times 10^{15}$ rad/s
      \item Si₃N₄: Five-oscillator model covering UV-visible-IR spectrum
    \end{itemize}
  
  \item \textbf{Metamaterials:} Engineered negative-index materials with tailored resonances
    \begin{itemize}
      \item Split-ring resonators: Tunable magnetic response at THz frequencies
      \item Wire arrays: Controlled plasma frequency for negative permittivity
    \end{itemize}
\end{itemize}

\subsubsection*{Parameter Presets}
Standard material presets have been implemented for rapid prototyping:

\begin{table}[h]
\centering
\caption{Drude-Lorentz Parameter Presets for Common Vacuum Engineering Materials}
\begin{tabular}{lccccc}
\toprule
\textbf{Material} & \textbf{$\varepsilon_\infty$} & \textbf{$\omega_p$ (rad/s)} & \textbf{$\gamma$ (rad/s)} & \textbf{$\omega_0$ (rad/s)} & \textbf{Application} \\
\midrule
Gold & 1.0 & $1.37 \times 10^{16}$ & $1.05 \times 10^{14}$ & 0 & Plasmonic cavities \\
Silver & 1.0 & $1.39 \times 10^{16}$ & $3.23 \times 10^{13}$ & 0 & Low-loss mirrors \\
SiO₂ & 2.25 & 0 & $1.0 \times 10^{13}$ & $2.75 \times 10^{15}$ & Dielectric spacers \\
Si₃N₄ & 4.0 & 0 & $5.0 \times 10^{12}$ & $3.5 \times 10^{15}$ & High-index layers \\
\bottomrule
\end{tabular}
\end{table}

\subsubsection*{Frequency-Dependent Optimization}
The permittivity models are optimized across three spectral regimes:
\begin{enumerate}
  \item \textbf{Optical (400-800 nm):} Primary focus for Casimir force applications
  \item \textbf{Near-infrared (0.8-3 μm):} Critical for thermal radiation management
  \item \textbf{Mid-infrared (3-30 μm):} Important for surface plasmon resonances
\end{enumerate}

Model validation is performed against experimental ellipsometry data with typical accuracy better than 2\% in the real part and 5\% in the imaginary part of the permittivity.

\end{document}
