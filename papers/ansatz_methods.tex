% ansatz_methods.tex
\documentclass[11pt]{article}
\usepackage{amsmath,amssymb}
\usepackage{graphicx}
\usepackage{hyperref}
\usepackage{booktabs}

\begin{document}

\section*{Optimization Ansätze for Warp Drive Shape Functions}

\subsection*{Overview}
This document presents various ansätze for optimizing the warp drive shape function $f(r)$ in the Alcubierre metric. The optimization targets include minimizing the total negative energy requirement while satisfying boundary conditions and quantum inequality constraints.

\subsection*{Single Gaussian Ansatz}
The baseline single-parameter Gaussian profile:
\[
  f(r) = A_1\exp\!\bigl[-\tfrac{(r-r_1)^2}{2\sigma_1^2}\bigr]
\]
provides a 3-parameter optimization space $(A_1, r_1, \sigma_1)$.

\subsection*{Multi-Gaussian Superposition}
The $n$-Gaussian ansatz extends the baseline through linear superposition:
\[
  f(r) = \sum_{i=1}^n A_i\exp\!\bigl[-\tfrac{(r-r_i)^2}{2\sigma_i^2}\bigr]
\]
yielding a $3n$-dimensional parameter space $(A_i, r_i, \sigma_i)_{i=1}^n$.

\subsection*{Hybrid Spline-Gaussian Ansatz}
Combines cubic spline core regions with Gaussian tails:
\[
  f(r) = \begin{cases}
    \text{cubic spline} & r \in [r_{\rm core}^{\rm min}, r_{\rm core}^{\rm max}] \\
    \sum_{i=1}^{n_G} A_i\exp\!\bigl[-\tfrac{(r-r_i)^2}{2\sigma_i^2}\bigr] & \text{otherwise}
  \end{cases}
\]

\subsection*{8-Gaussian Two-Stage Ansatz}
The 26-dimensional ansatz
\[
  f(r)=\sum_{i=1}^8 A_i\exp\!\bigl[-\tfrac{(r-r_i)^2}{2\sigma_i^2}\bigr]
\]
is optimized via CMA-ES global search (4,800 evals), L-BFGS-B refinement, and JAX-accelerated local optimization, achieving $E_-=-1.48\times10^{53}\,\mathrm{J}$.

\subsection*{Ultimate B-Spline Control-Point Ansatz}
We replace Gaussian superpositions with a linear B-spline over \(N\) control points, jointly optimizing \((\mu,G_{\rm geo},\{c_i\})\) via CMA-ES → JAX, enforcing hard stability penalties and surrogate-assisted exploration. This yields \(E_-<2.0\times10^{54}\,\mathrm{J}\).

\subsection*{Optimization Strategy Hierarchy}
\begin{enumerate}
\item \textbf{Global Search:} CMA-ES (Covariance Matrix Adaptation Evolution Strategy) for robust exploration of high-dimensional parameter spaces
\item \textbf{Local Refinement:} L-BFGS-B for gradient-based convergence to local optima
\item \textbf{Acceleration:} JAX automatic differentiation for fast gradient computation
\item \textbf{Validation:} Constraint verification and stability analysis
\end{enumerate}

\subsection*{Boundary Conditions and Constraints}
All ansätze must satisfy:
\begin{itemize}
\item Asymptotic behavior: $f(r) \to 0$ as $r \to \infty$
\item Smoothness: $f(r) \in C^2(\mathbb{R})$
\item Energy constraints: Total negative energy $E_- < 0$
\item Quantum inequality compliance: $\int \rho_{\rm eff}(t)f(t)dt \geq \text{bound}$
\end{itemize}

\subsection*{Polymer-Matter Hamiltonian \& Parameter Sweep}

The polymer quantization approach extends traditional field theory to include discrete quantum geometry effects through modified canonical commutation relations. The polymer-quantized matter Hamiltonian incorporates these corrections through trigonometric substitutions:

\[
H_{\text{matter}} = \int \left[ \pi^2 \text{sinc}^2(\mu \pi) + (\nabla \phi)^2 + m^2 \phi^2 \right] d^3r
\]

where $\text{sinc}(\mu \pi) = \sin(\mu \pi)/(\mu \pi)$ represents the polymer correction with scale parameter $\mu$.

\subsubsection*{Curvature-Matter Coupling}

The nonminimal curvature-matter interaction enables spacetime-driven particle creation:
\[
H_{\text{int}} = \lambda \int \sqrt{f} R \phi^2 d^3r
\]
where $\lambda$ is the coupling strength, $f$ is the metric function, and $R$ is the Ricci scalar computed using discrete finite-difference methods.

\subsubsection*{Discrete Ricci Scalar Formula}

For numerical implementation, the Ricci scalar is computed using stable finite difference schemes:
\[
R_i = -\frac{f''_i}{2f_i^2} + \frac{(f'_i)^2}{4f_i^3}
\]
where derivatives are computed using central differences with regularization near $f \approx 0$.

\subsubsection*{Multi-Objective Optimization}

The complete optimization objective balances matter creation against constraint violations:
\[
J = \Delta N - \gamma A - \kappa C
\]
where:
\begin{itemize}
\item $\Delta N = \int_0^T 2\lambda \sum_i R_i \phi_i \pi_i \, dt$ is the matter creation integral
\item $A = \int_0^T \sum_i |G_{tt,i} - 8\pi(T_{m,i} + T_{\text{int},i})| \, dt$ is the constraint anomaly
\item $C = \int_0^T \sum_i |R_i| \, dt$ is the curvature cost
\item $\gamma = 1.0$ and $\kappa = 0.1$ are penalty weights
\end{itemize}

\subsubsection*{Parameter Sweep Methodology}

Systematic exploration of the four-dimensional parameter space:
\begin{align}
\lambda &\in [0.005, 0.02] \quad \text{(curvature-matter coupling)} \\
\mu &\in [0.15, 0.25] \quad \text{(polymer scale parameter)} \\
\alpha &\in [1, 3] \quad \text{(metric enhancement amplitude)} \\
R_0 &\in [1, 2] \quad \text{(characteristic bubble radius)}
\end{align}

The refined 54-point sweep around optimal configuration $\{\lambda=0.01, \mu=0.20, \alpha=2.0, R_0=1.0\}$ validated the near-zero creation regime and confirmed robustness of the optimal parameter window.

\end{document}
