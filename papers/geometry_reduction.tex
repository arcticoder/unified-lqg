% geometry_reduction.tex
\documentclass[11pt]{article}
\usepackage{amsmath,amssymb}
\usepackage{hyperref}

\begin{document}

\section*{Geometric Energy Reduction Analysis}

\section{Van den Broeck–Natário Geometric Reduction}
The hybrid profile induces
\[
  \mathcal{R}_{\rm geo} 
  = \Bigl(\frac{R_{\rm ext}}{R_{\rm int}}\Bigr)^3 
  \;\approx\; 10^{-5}\text{–}10^{-6},
\]
so that
\[
  E_{\rm required}^{\rm VdB} 
  = E_{\rm required}^{\rm Alcubierre} \times \mathcal{R}_{\rm geo}.
\]
Numerically, this is a $10^5$–$10^6\times$ reduction.

\subsection*{Geometric Profile Design}
The Van den Broeck–Natário approach employs a carefully engineered metric of the form:
\[
  ds^2 = -dt^2 + (dx - v_s f(r_s) dt)^2 + dy^2 + dz^2
\]
where the shape function $f(r_s)$ transitions between different geometric regimes:
\begin{itemize}
\item \textbf{Internal region} ($r < R_{\rm int}$): Flat spacetime with passenger compartment
\item \textbf{Transition region} ($R_{\rm int} < r < R_{\rm ext}$): Rapid metric variation containing the warp field
\item \textbf{External region} ($r > R_{\rm ext}$): Asymptotically flat spacetime
\end{itemize}

\subsection*{Energy Scaling}
The total energy requirement scales as:
\[
  E_{\rm total} \propto R_{\rm int}^3 \times \left(\frac{R_{\rm int}}{R_{\rm ext}}\right)^3
\]

By making $R_{\rm ext} \gg R_{\rm int}$, the geometric factor $\mathcal{R}_{\rm geo} = (R_{\rm int}/R_{\rm ext})^3$ becomes arbitrarily small, dramatically reducing energy requirements.

\end{document}
