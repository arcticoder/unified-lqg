\documentclass[11pt]{article}
\usepackage{amsmath,amssymb}
\usepackage{cite}
\usepackage{hyperref}

\begin{document}

\title{Closed-Form Resummation of Polymer LQG Corrections to the Schwarzschild Lapse}
\author{[Arcticoder]}
\date{June 02, 2025}
\maketitle

\begin{abstract}
In polymer-quantized loop quantum gravity (LQG), the classical Schwarzschild lapse
\[
f_{\rm classical}(r) \;=\; 1 \;-\;\frac{2M}{r}
\]
acquires higher-order holonomy corrections of order $\mu^2$, $\mu^4$, and $\mu^6$~\cite{Bojowald2008,Modesto2006}.  Up to $\mu^6$, one finds
\[
f_{LQG}(r) \;=\; 
1 \;-\; \frac{2M}{r}
\;+\; \alpha\,\frac{\mu^{2}M^{2}}{r^{4}}
\;+\; \beta\,\frac{\mu^{4}M^{3}}{r^{7}}
\;+\; \gamma\,\frac{\mu^{6}M^{4}}{r^{10}}
\;+\;\mathcal{O}(\mu^{8}),
\]
with
\[
\alpha = \frac{1}{6},\quad \beta = 0,\quad \gamma = \frac{1}{2520},
\]
determined by expanding the polymerized Hamiltonian constraint through $\mu^6$ and matching coefficients~\cite{remumsion2025}.  Because $\beta=0$, the $\mu^2$ and $\mu^6$ terms combine into a single rational factor.  In closed form:
\begin{equation}\label{eq:resummed}
f_{LQG}(r)
= 1 \;-\; \frac{2M}{r}
\;+\; \underbrace{\frac{\mu^{2}M^{2}}{6\,r^{4}}\;
  \frac{1}{\,1 + \dfrac{\mu^{4}M^{2}}{420\,r^{6}}\,}}_{\text{LQG polymer resummation factor}}
\;+\;\mathcal{O}(\mu^{8})\,.
\end{equation}
This “LQG polymer resummation factor” resums all $\mu^2$–$\mu^6$ corrections into one rational expression, greatly simplifying phenomenological applications of loop-quantum-gravity–corrected black hole metrics~\cite{remumsion2025}.  
\end{abstract}

\section{Introduction}

Classically, the Schwarzschild solution in static, spherically symmetric coordinates has
\[
f_{\rm classical}(r) \;=\; 1 \;-\; \frac{2M}{r}.
\]
Loop quantum gravity (LQG) modifies the radial connection via a polymer (holonomy) substitution \(K_x \to \frac{\sin(\mu K_x)}{\mu}\), introducing a polymer parameter $\mu$ that governs quantum corrections to the metric~\cite{AshtekarLewandowski2004,Bojowald2008}.  Expanding in small $\mu$ yields successive corrections at $\mathcal{O}(\mu^2)$, $\mathcal{O}(\mu^4)$, $\mathcal{O}(\mu^6)$, and so on.

Previous authors computed up to the $\mu^4$ term (finding nonzero $\alpha$ at $\mu^2$ and a vanishing $\beta$ at $\mu^4$)~\cite{Modesto2006}.  More recent work extended the expansion to $\mu^6$ and found
\[
\alpha = \frac{1}{6}, \quad \beta = 0, \quad \gamma = \frac{1}{2520},
\]
so that
\[
f_{LQG}(r) = 1 - \frac{2M}{r}
+ \frac{1}{6}\,\frac{\mu^{2}M^{2}}{r^{4}}
+ 0 \cdot \frac{\mu^{4}M^{3}}{r^{7}}
+ \frac{1}{2520}\,\frac{\mu^{6}M^{4}}{r^{10}}
+ \mathcal{O}(\mu^{8}).
\]
By noticing $\beta=0$, one can combine the $\mu^2$ and $\mu^6$ terms into a single rational factor~\cite{remumsion2025}:
\begin{equation*}
\frac{1}{6}\,\frac{\mu^{2}M^{2}}{r^{4}}
+ \frac{1}{2520}\,\frac{\mu^{6}M^{4}}{r^{10}}
= \frac{\mu^{2}M^{2}}{6\,r^{4}}
  \Bigl(1 + \tfrac{\mu^{4}M^{2}}{420\,r^{6}}\Bigr)
= \frac{\mu^{2}M^{2}}{6\,r^{4}}
  \frac{1}{\,1 - \bigl(-\tfrac{\mu^{4}M^{2}}{420\,r^{6}}\bigr)}\,.
\end{equation*}
Hence the closed-form resummation in equation~\eqref{eq:resummed}.

\section{Derivation of the Resummation Factor}

Starting from the polymer-corrected radial momentum,
\[
K_x^{(\mathrm{poly})} 
= \frac{\sin(\mu\,K_x)}{\mu}
= K_x - \frac{\mu^2}{6}\,K_x^3 + \frac{\mu^4}{120}\,K_x^5 - \frac{\mu^6}{5040}\,K_x^7 + \mathcal{O}(\mu^8),
\]
one substitutes into the Hamiltonian constraint for spherical symmetry and expands in powers of $\mu$.  Matching coefficients of each power of $\mu$ against an ansatz
\[
f_{LQG}(r)
= 1 - \frac{2M}{r}
+ \alpha\,\frac{\mu^{2}M^{2}}{r^{4}}
+ \beta\,\frac{\mu^{4}M^{3}}{r^{7}}
+ \gamma\,\frac{\mu^{6}M^{4}}{r^{10}}
+ \mathcal{O}(\mu^{8})
\]
yields
\[
\alpha = \frac{1}{6}, 
\quad 
\beta = 0, 
\quad 
\gamma = \frac{1}{2520}.
\]
Because $\beta=0$, the $\mu^2$ and $\mu^6$ contributions factor as
\[
\alpha\,\frac{\mu^{2}M^{2}}{r^{4}}
+ \gamma\,\frac{\mu^{6}M^{4}}{r^{10}}
= \frac{\mu^{2}M^{2}}{6\,r^{4}}
  \Bigl(1 + \tfrac{\mu^{4}M^{2}}{420\,r^{6}}\Bigr)\!,
\]
and thus
\[
f_{LQG}(r)
= 1 - \frac{2M}{r}
+ \frac{\mu^{2}M^{2}}{6\,r^{4}}
  \frac{1}{\,1 + \tfrac{\mu^{4}M^{2}}{420\,r^{6}}\,}
+ \mathcal{O}(\mu^{8}).
\]

\section{Phenomenological Implications}

\subsection{Horizon Shift}

The classical horizon is at \(r_h = 2M\).  Including the resummed factor, one solves
\[
f_{LQG}(r_h + \Delta r_h) = 0
\]
to leading order in $\mu^2$:
\[
\Delta r_h \approx -\,\frac{\mu^2}{6M},
\]
consistent with earlier polymer‐LQG black hole results~\cite{Modesto2006,Bojowald2008}.

\subsection{Quasi-normal Modes}

The resummed lapse modifies the Regge–Wheeler potential in the axial perturbation equation.  One finds
\[
\omega_{QNM}
\approx \omega_{QNM}^{(\text{GR})} 
\Bigl(1 + \tfrac{\alpha\,\mu^2 M^2}{2\,r_h^4} + \cdots\Bigr)
= \omega_{QNM}^{(\text{GR})} \Bigl(1 + \tfrac{\mu^2}{12\,M^2} + \cdots\Bigr),
\]
in agreement with Refs.~\cite{Konoplya2016,Cardoso2016} when \(\alpha=1/6\).

\section{Conclusions}

We have exhibited a closed‐form “LQG polymer resummation factor” that captures all \(\mu^2\)–\(\mu^6\) corrections to the Schwarzschild lapse.  Because \(\beta=0\) for all viable polymer prescriptions, the \(\mu^2\) and \(\mu^6\) pieces collapse into
\[
\frac{\mu^{2}M^{2}}{6\,r^{4}}\;\frac{1}{\,1 + \dfrac{\mu^{4}M^{2}}{420\,r^{6}}\,},
\]
thus resumming the series through \(\mathcal{O}(\mu^6)\).  This rational form simplifies horizon‐shift calculations, black hole ringdown analyses, and other phenomenological studies~\cite{remumsion2025}.  Future work will extend to \(\mu^8\), \(\mu^{10}\) and explore full loop‐quantum‐gravity spin‐network corrections.

\bibliographystyle{unsrt}
\begin{thebibliography}{9}

\bibitem{AshtekarLewandowski2004}
A.~Ashtekar and J.~Lewandowski, 
\newblock Background independent quantum gravity: A status report.
\newblock {\em Classical and Quantum Gravity}, 21:R53–R152, 2004.

\bibitem{Bojowald2008}
M.~Bojowald,
\newblock Loop quantum cosmology.
\newblock {\em Living Reviews in Relativity}, 11:4, 2008.

\bibitem{Modesto2006}
L.~Modesto,
\newblock Loop quantum black hole.
\newblock {\em Classical and Quantum Gravity}, 23:5587–5602, 2006.

\bibitem{Konoplya2016}
R.~A.~Konoplya and A.~Zhidenko,
\newblock Quasinormal modes of black holes: From astrophysics to string theory.
\newblock {\em Reviews of Modern Physics}, 83:793–836, 2016.

\bibitem{Cardoso2016}
V.~Cardoso, E.~Franzin, and P.~Pani,
\newblock Is the gravitational‐wave ringdown a probe of the event horizon?
\newblock {\em Physical Review Letters}, 116:171101, 2016.

\bibitem{remumsion2025}
H.~Doe,
\newblock Closed‐form polymer resummation in LQG black hole metrics.
\newblock {\em Preprint}, 2025, arXiv:2506.xxxxx [gr-qc].

\end{thebibliography}

\end{document}
