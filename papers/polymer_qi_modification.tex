% polymer_qi_modification.tex
\documentclass[11pt]{article}
\usepackage{amsmath,amssymb}
\usepackage{hyperref}

\begin{document}

\section*{Polymer‐Modified Quantum Inequality (Corrected)}

\subsection*{Classical Ford-Roman Bound}
The standard quantum inequality for a massless scalar field states:
\[
  \int_{-\infty}^{\infty} \langle T_{00}(x,t) \rangle f(t)\,dt \geq -\frac{C}{\tau^2},
\]
where $f(t)$ is a normalized sampling function with characteristic width $\tau$, and $C$ is a numerical constant.

\section{Polymer‐Modified Quantum Inequality (Corrected)}
\[
  \int_{-\infty}^\infty \rho_{\rm eff}(t)\,f(t)\,dt 
  \;\ge\; -\,\frac{\hbar\,\sinc(\pi\mu)}{12\pi\,\tau^2}, 
  \quad \sinc(\pi\mu)=\frac{\sin(\pi\mu)}{\pi\mu}.
\]

\medskip
\noindent\textbf{Numerical Verification (No False Positives).}
Scanning $\mu\in[10^{-8},10^{-4}]$ with $f(t)=e^{-t^2/(2\tau^2)}/(\sqrt{2\pi}\,\tau)$ shows
\[
  \int_{-\infty}^\infty \rho_{\rm eff}(t)\,f(t)\,dt \;<\; 0
  \quad (\forall\,\mu>0),
\]
confirming that $\sinc(\pi\mu)$ incurs no spurious violations.

\subsection*{Polymer Quantization Modification}
Loop quantum gravity introduces polymer quantization through the replacement:
\[
  \hat{p}_i \rightarrow \hat{\pi}_i^{\rm poly} = \frac{\sin(\mu\,\hat{p}_i)}{\mu},
\]
where $\mu$ is the fundamental polymer scale. This modification alters the commutation relations and subsequently modifies the quantum inequality bound through the $\sinc(\pi\mu)$ factor.

\subsection*{Physical Interpretation}
The $\sinc(\pi\mu)$ factor represents the quantum geometry's influence on negative energy bounds. For small $\mu$, $\sinc(\pi\mu) \approx 1$ and we recover the classical bound. For finite $\mu$, the oscillatory nature of the sinc function provides relaxation opportunities at specific polymer scales.

\end{document}
